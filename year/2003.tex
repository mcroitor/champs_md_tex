\chapter{Год 2003}
\textbf{Судья: Залокоцкий Роман (Украина)}

\begin{tabularx}{\textwidth}{l l r}
Место & Участник & Баллы \\
\hline
1 место & Чебанов Николай & 5 \\
2 место & Иванов Альберт & 4,5 \\
3 место & Кожокарь Вячеслав & 4 \\
4 место & Мушет Ион & 3,7 \\
5 место & Кройтор Михаил & 3,5 \\
\end{tabularx}

\begin{center} 
 \begin{tabular}{ c c }
\textbf{№ 24. Чебанов Николай} & \textbf{№ 25. Иванов Альберт} \\
\small{Уральский Проблемист, 2002} & \small{}\\
\small{} & \small{}\\
\chessboard[
\diagramsize,
setfen=8/8/1r6/1pk3r1/2pp4/pnbb4/Np1pq3/7K,
label=false,
showmover=false] & 
\chessboard[
\diagramsize,
setfen=8/pp6/qr2P3/P5R1/3n1p2/6B1/6PP/R3K1kn,
label=false,
showmover=false] \\
\textbf{H\mate{}4 \knight{}a2 $\to$ \rook{}a2 $\to$ \bishop{}a2} & \textbf{H\mate{}2*   2 решения} 
 \end{tabular}
\end{center}

№ 24 \begin{description} 
\item [a)] original \textbf{1.\bishop{}c2 \knight{}xc3 2.\queen{}d3 \knight{}e2 3.\king{}b4 \knight{}c1 4.\king{}c3 \knight{}a2 \mate}
\item [b)] wRa2  \textbf{1.b4 \rook{}xa3 2.\king{}b5 \rook{}xb3 3.\king{}a5 \rook{}xb2 4.\rook{}b5 \rook{}a2 \mate}
\item [c)] wBa2  \textbf{1.\bishop{}b4 \bishop{}xb3 2.c3 \bishop{}c2 3.\king{}c4 \bishop{}b1 4.\rook{}c5 \bishop{}a2 \mate}
\end{description}

№ 25 Иллюзорная игра \textbf{1. ... \bishop{}xf4! 2. \knight{}b5 \king{}e2 \mate}
 
Ложный след \textbf{1. \rook{}b2? \rook{}b5 2. \rook{}xg2 O-O-O \mate}

\begin{enumerate*}[label={\alph*)}] 
\item \textbf{1. f3 \rook{}b5 2. fxg2 \king{}d2/c1 \mate}
\item \textbf{1. \knight{}f2 \rook{}b5 2. \king{}h1 \king{}xf2 \mate}
\end{enumerate*}

\begin{center} 
 \begin{tabular}{ c c }
\textbf{№ 26. Кожокарь Вячеслав} & \textbf{№ 27. Мушет Ион} \\
\small{} & \small{}\\
\small{} & \small{}\\
\chessboard[
\diagramsize,
setfen=8/B7/5P2/8/2k5/8/3PpK2/5br1,
label=false,
showmover=false] & 
\chessboard[
\diagramsize,
setfen=2NN4/1p1K4/1r6/3k4/8/8/8/8,
label=false,
showmover=false] \\
\textbf{=} & \textbf{H\mate{}3   +B\bishop{}a2} 
 \end{tabular}
\end{center}

№ 26 \textbf{1. f7 \rook{}h1!} 
(1... e1=\queen{}+ 2. \king{}xe1 \rook{}h1 3.\bishop{}g1! \rook{}h8 (3... \rook{}xg1 4. \king{}f2!) 4. \king{}xf1) 
\textbf{2. f8=\queen{} e1=\queen{}+ 3. \queen{}xe1 \bishop{}d3+ 4. \bishop{}g1! \rook{}xg1+ 5. \king{}f2 \rook{}f1+ 6. \king{}e3 \rook{}xf8} пат

№ 27 \begin{description} 
\item [a)] original textbf{1. \rook{}b4 \king{}c7 2. b5 \knight{}b6 3. \king{}c5 \knight{}e6\mate}
\item [b)] +B\bishop{}a2 \textbf{1. \king{}c5 \king{}e7 2. \bishop{}d5 \knight{}e6 3. \king{}c6 \knight{}a7\mate}
\end{description}

\begin{wrapfigure}{r}{0.5\textwidth}
\begin{center} 
 \begin{tabular}{ c }
\textbf{№ 28. Кройтор Михаил} \\
\small{} \\
\small{} \\
\chessboard[
\diagramsize,
setfen=8/5N2/2K4B/8/3k4/5pP1/2Q5/3rb3,
label=false,
showmover=false] \\
\textbf{\mate{}2} 
 \end{tabular}
\end{center}
\end{wrapfigure}

№ 28 Задача популярного стиля \textbf{1. \knight{}g5! $\sim$ 2. \knight{}xf3 \mate;}
 \textbf{1. ... \king{}e3 2. \knight{}e6\mate;}
 \textbf{1. ... \king{}e5 2. \bishop{}g7\mate;}
 \textbf{1. ... \rook{}d3 2. \queen{}c5\mate;}
 \textbf{1. ... \bishop{}c3 2. \queen{}e4\mate}