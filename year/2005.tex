\chapter{Год 2005}
\textbf{Судья: Усманов Рашид (Россия)}

\begin{tabularx}{\textwidth}{l l r}
Место & Участник & Баллы \\
\hline
1 место & Иванов Альберт & 4,8 \\
2 место & Чебанов Николай & 4,7 \\
3 место & Гинда Анатолий & 4 \\
4 место & Ткачук Владимир & 1,5 \\
\end{tabularx}

\begin{center} 
 \begin{tabular}{ c c }
\textbf{№ 33. Иванов Альберт} & \textbf{№ 34. Чебанов Николай} \\
\small{} & \small{}\\
\small{} & \small{}\\
\chessboard[
\diagramsize,
setfen=7B/n5n1/1p6/1p2rp2/1K2Qrb1/PP3p2/BkpR1p2/q3bN2,
label=false,
showmover=false] & 
\chessboard[
\diagramsize,
setfen=4N3/pp2p1KB/qk2P3/bp2R3/1p6/1P6/1P4P1/8,
label=false,
showmover=false] \\
\textbf{H\mate{}2   2 варианта} & \textbf{\mate{}12} 
 \end{tabular}
\end{center}

№ 33. \begin{enumerate*}[label={\alph*)}] 
\item \textbf{1. \knight{}e6 \bishop{}xe5 2. \knight{}d4 \queen{}xc2\mate}
\item \textbf{1. \rook{}c5 \bishop{}xg7 2. \rook{}c3 \rook{}xc2\mate}
\end{enumerate*}

№ 34. 

\begin{center} 
 \begin{tabular}{ c c }
\textbf{№ 35. Гинда Анатолий} & \textbf{№ 36. Ткачук Владимир} \\
\small{№.4452, Buletin Problemistic, 2005} & \small{Buletin Problemistic, 2005}\\
\small{} & \small{}\\
\chessboard[
\diagramsize,
setfen=8/1q6/3Nr3/3PkB2/4p1b1/2K5/8/8,
label=false,
showmover=false] & 
\chessboard[
\diagramsize,
setfen=1b6/2r4R/3KPPn1/5kP1/2r5/3P2Pp/3Q2B1/6RN,
label=false,
showmover=false] \\
\textbf{H\mate{}3   3 варианта} & \textbf{S\mate{}5} 
 \end{tabular}
\end{center}

№ 35 \begin{enumerate*}[label={\alph*)}] 
\item \textbf{1.\king{}xd5 \knight{}xe4 2.\rook{}e5 \bishop{}c8 3.\bishop{}e6 \bishop{}xb7\mate} 
\item \textbf{1.\queen{}f7 \bishop{}xe4 2.\queen{}f4 \king{}d3 3.\rook{}f6 \knight{}c4\mate}
\end{enumerate*}

№ 36 1.g4+! 
  1...\rook{}xg4 2.\bishop{}e4+ \rook{}xe4 3.\knight{}g3+ \king{}g4 4.\knight{}e2+ \king{}f3/Kf5 5.\knight{}d4+ \rook{}xd4\mate 
  1...\king{}xg4 2.\bishop{}d5+ \king{}f5 3.\knight{}g3+ \king{}g4 4.\knight{}h5+ и 4...\king{}f5 5.\knight{}g7+ \rook{}xg7\mate или 4...\king{}h4 5.\knight{}f4+ \rook{}xh7\mate