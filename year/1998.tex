\chapter{Год 1998}
\textbf{Судья: Великий Николай (Украина)}

\begin{tabularx}{\textwidth}{l l r}
Место & Участник & Баллы \\
\hline
1 место & Иванов Альберт & 4,7 \\
2 место & Молдовяну Пётр & 4,6 \\
3 место & Молдовяну Николай & 4,5 \\
4 место & Воронов Анатолий & 4 \\
5 место & Гинда Анатолий & 3,5 \\
\end{tabularx}

\bigskip

\begin{center} 
 \begin{tabular}{ c c }
\textbf{№ 1. Иванов Альберт} & \textbf{№ 2. Молдовяну Пётр} \\
\small{VIII командное пер-во СССР} & \small{Problemblad} \\
\small{1972} & \small{2006} \\
\chessboard[
\diagramsize,
setfen=8/3nPr2/1p2K3/p1b3P1/RBpPk1pn/pBq2p2/pP6/3b4,
label=false,
showmover=false] & 
\chessboard[
\diagramsize,
setfen=3b1q2/K1p1Bp2/7p/8/2pkN3/3b2P1/4p3/8,
label=false,
showmover=false]\\
\textbf{H\mate{}2   2 решения} & \textbf{H\mate{}3   2 решения} 
 \end{tabular}
\end{center}

№1. Эта позиция есть исправление задачи, участвовавшей в VIII командном первенстве СССР. В задаче представлена тема Умнова в 2 вариантах: Черная фигура освобождает поле для другой черной фигуры. В первом варианте ферзь освобождает поле для пешки, во втором наоборот: 
\textbf{1. \queen{}d3 \bishop{}d2 2. c3 \bishop{}d5\mate{}}, Второе решение исходит из предположения, что возможно взятие на проходе: \textbf{1.cxd3 \bishop{}xc5 2.\queen{}xc4 \rook{}xc4\mate{}}.

№2 В обоих решениях проходит бело-черная прокладка пути -- вариация бристольской темы: \textbf{1. c5 \bishop{}h4 2. \bishop{}g5 \knight{}c3 3. \bishop{}e3 \bishop{}f6\mate{}; 1. \king{}e3 \bishop{}a3 2. \queen{}b4 \knight{}g5 3. \queen{}d2 \bishop{}c5\mate{}}

\begin{center} 
 \begin{tabular}{ c c }
\textbf{№ 3. Молдовяну Николай} & \textbf{№ 4. Воронов Анатолий} \\
\small{Problem} & \small{XIV командное пер-во СССР} \\
\small{1970} & \small{1988} \\
\chessboard[
\diagramsize,
setfen=B7/3p1N2/4P3/2P5/3B3p/6pr/6Pp/4K2k,
label=false,
showmover=false] & 
\chessboard[
\diagramsize,
setfen=2k5/2p3p1/2P3pb/1R3p2/1B3p2/K1N4n/8/7B,
label=false,
showmover=false] \\
\textbf{\mate{}3} & \textbf{\mate{}4} 
 \end{tabular}
\end{center}

№3 \textbf{1. c6!} Угроза \textbf{2. exd7 ZZ \king{}xg2 3. c7\mate{}; 1. ... dxe6/ dxc6/ d6/ d5 2. \knight{}e5/ \bishop{}c5/ \knight{}xd6/ c7 ZZ \king{}xg2 3. c7/ \bishop{}xc6/ c7/ \bishop{}xd5\mate{}}. Реализован механизм Пикенини -- четырехкратная игра черной пешки.

№4 Задача представляет собой взаимную обструкцию -- за белых в попытках и за черных в решении:\textbf{1.\bishop{}d5? \knight{}g5!; 1.\knight{}d5? \bishop{}g5!; 1.\bishop{}f3!} zugzwang. \textbf{1. ... \knight{}g5 2.\knight{}d5! $\sim$ 3.\knight{}e7+ \king{}d8 4.\rook{}b8\mate{}; 1. ... \bishop{}g5 2.\bishop{}d5 $\sim$ 3.\bishop{}e6+ \king{}d8 4.\rook{}b8\mate{}}.

\begin{wrapfigure}{r}{0.5\textwidth}
\begin{center} 
 \begin{tabular}{ c }
\textbf{№ 5. Гинда Анатолий} \\
\chessboard[
\diagramsize,
setfen=3B4/4n3/8/1Rb5/3k4/1K6/8/8,
label=false,
showmover=false] \\
\textbf{H\mate{}3} 
 \end{tabular}
\end{center}
\end{wrapfigure}

№5 \textbf{1. \bishop{}d6 \rook{}h5 2. \knight{}c6 \bishop{}g5 3.\king{}c5 \bishop{}e3\mate{}}.
