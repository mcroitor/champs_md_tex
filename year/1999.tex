\chapter{Год 1999}
\textbf{Судья: Рейцен Евгений (Украина)}

\begin{tabularx}{\textwidth}{l l r}
Место & Участник & Баллы \\
\hline
1 место & Иванов Альберт & 5 \\
2 место & Молдовяну Пётр & 4 \\
3 место & Чиканов Николай & 2,9 \\
\end{tabularx}

\bigskip

\begin{center} 
 \begin{tabular}{ c c }
\textbf{№ 6. Иванов Альберт} & \textbf{№ 7. Молдовяну Пётр} \\
\small{III командное пер-во Украины} & \small{ }\\
\small{1972} & \small{ }\\
\chessboard[
\diagramsize,
setfen=k1K5/P1pp2p1/pP1Pp3/1p1P1p2/4PP2/2PP4/2p5/8,
label=false,
showmover=false] & 
\chessboard[
\diagramsize,
setfen=8/2p5/2R4B/K1Nn4/2NkP3/R5bQ/3pn3/1b3Br1,
label=false,
showmover=false] \\
\textbf{+} & \textbf{\mate{}2} 
 \end{tabular}
\end{center}

№6 Этюд на ретроанализ -- сначала необходимо выяснить, чья очередь хода. Как видно, на доске все пешки. Белые пешки d5 и d6 пришли с полей g2 и h2 взяв 3 и 4 фигуры соответственно всего 7. Баланс черных – (9 фигур на доске) 9+7 = 16 – больше взятий белые делать не могли. Значит, черная пешка а6 пришла с поля b7, b5 -- с линии а (поля а6 или а7) – 2 взятия. Пешка с2 попала с поля h7, взяв 5 фигур, или с линии f, взяв 3 фигуры, а пешка f5 – с поля h7, взяв 2 фигуры. Баланс белых фигур: 9+2+5=16.

Каким был последний ход черных? Если 0. … \bishop{}a, то черный слон был взят на поле с8, значит у черных на 1 фигуру больше – не сходится баланс фигур. По этой же причине не мог быть ход е6: чернопольный слон тогда был бы взят на поле f8. Следует заметить, что конфигурация пешек a7, a6, b6 и b5 возможна только при взятии черными с полей a6 и b7 – белые поля. Если считать, что последний ход был 0. … f7(f6)-f5, то пешка с2 взяла еще 5 фигур, стоявших на белых полях. Но один из слонов – чернопольный, значит, опять не сходится баланс фигур.
1. ... cxb6 2.\king{}xd7 exd5 3.e5 c1=\queen{} 4.e6 \queen{}xc3 5.e7 \king{}xa7 6.e8=\queen{} с победой.

№7 В ложном следе, после 1. \queen{}d7? (угроза 2. \queen{}xd5\mate{}) на ходы коня е2 грозят маты слоном: \textbf{1. ... \knight{}ec3 [a] 2.\bishop{}e3 [A] \mate{}; 1. ... \knight{}ef4 [b] 2.\bishop{}g7 [B] \mate{}; (1. ... \bishop{}xe4 2.\knight{}e6 \mate{})}. Опровержение -- \textbf{1. ... \bishop{}d6!}. 

В решении на эти защиты черных маты меняются, сами же маты проходят на другие защиты (полная форма темы Рухлиса): \textbf{1. \queen{}e6! ~ 2.\queen{}xd5\mate{}; 1. ... \knight{}ef4 [a] 2.\queen{}e5 [C] \mate{}; 1. ... \knight{}ec3 [b] 2.\knight{}b3 [D] \mate{}; 1. ... \knight{}~2.\bishop{}(x)e3 [A] \mate{}; 1. ... \knight{}df4 2.\bishop{}g7 [B] \mate{}; (1... \king{}xc4 2.\queen{}xd5\mate{}; 1... \bishop{}xe4 2.\queen{}xe4\mate{})}

Также в задаче есть черная коррекция в одном варианте. при желании можно увидеть и перемену матов на иллюзорную игру -- в начальной позиции на ходы \textbf{1... \king{}xc4/ \bishop{}xe4 2.\knight{}e6\mate{}}.

\begin{center} 
 \begin{tabular}{ c c }
\textbf{№ 8. Иванов Альберт} & \textbf{№ 9. Чиканов Николай} \\
\small{ } & \small{ }\\
\small{ } & \small{ }\\
\chessboard[
\diagramsize,
setfen=3k4/8/8/8/5p2/4b1p1/3p1Qp1/6Kb,
label=false,
showmover=false] & 
\chessboard[
\diagramsize,
setfen=3R4/p7/P7/K3p3/nPkP4/rpb1RnQ1/P1Pq4/3N3B,
label=false,
showmover=false] \\
\textbf{H\mate{}4} & \textbf{S\mate{}2} 
 \end{tabular}
\end{center}

№8 \textbf{1.\bishop{}b6 \queen{}e3 2.\bishop{}c7 \queen{}xf4 3.d1=\rook{}+ \queen{}f1 4.\rook{}d7 \queen{}f8\mate{}}. Реализовано двухкратное развязывание белого ферзя.

№9 После вступительного \textbf{1.\queen{}g4!} грозит \textbf{2.\queen{}c8 \knight{}c5\mate{}}. Черные в трех вариантах связывают свои фигуры в надежде на их дальнейшее развязывание (защита Шифмана в 3 вариантах): \textbf{1. ... \queen{}xd4 2. \rook{}xc3 \knight{}xc3\mate{}; 1. ... \knight{}xd4 2. \rook{}c8 \knight{}c5\mate{}; 1. ... \bishop{}xd4 2. \knight{}b2 \knight{}xb2\mate{}}
