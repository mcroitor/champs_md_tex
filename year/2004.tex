\chapter{Год 2004}
\textbf{Судья: Шаньшин Валерий (Украина)}

\begin{tabularx}{\textwidth}{l l r}
Место & Участник & Баллы \\
\hline
1 место & Иванов Альберт & 4,1 \\
2 место & Кройтор Михаил & 4 \\
3 место & Мушет Григорий & 3 \\
4 место & Димов Дмитрий & 2 \\
\end{tabularx}

\begin{center} 
 \begin{tabular}{ c c }
\textbf{№ 29. Иванов Альберт} & \textbf{№ 30. Кройтор Михаил} \\
\small{VI WCCT} & \small{}\\
\small{} & \small{}\\
\chessboard[
\diagramsize,
setfen=3b1n1Q/1NP3P1/5kp1/p2P4/1n1P1ppP/3PBB1P/K3N2b/8,
label=false,
showmover=false] & 
\chessboard[
\diagramsize,
setfen=8/8/3B4/3p2N1/3k2b1/2pB4/2P2K2/8,
label=false,
showmover=false] \\
\textbf{\mate{}3 см. текст} & \textbf{\mate{}3} 
 \end{tabular}
\end{center}

№ 29

№ 30 Задача является логической популярной. При ходе черных видно, что они в цугцванге: \textbf{1... \bishop{}~ 2.\knight{}e6/f3\mate}. Однако передать очередь хода не так просто. Ход белым слоном с d3, скажем, на a6 ( 1. \bishop{}a6 ?!), натыкается на сильное возражение (1... \bishop{}e2! ), после чего взятия черного слона приводит к пату.

Первым ходом слон выполняет критический ход, -- прячется на f1: \textbf{ 1.\bishop{}f1!}, -- с угрозой возврата назад и получением начальной позиции при ходе черных (2.\bishop{}d3). Попытка играть черным слоном приводит к тому, что белый конь получает возможность шаховать, и получаются 2 варианта: \textbf{ 1... \bishop{}d1(f3,h5) 2.\knight{}e6+ \king{}e4 3.\bishop{}d3\mate; 1... \bishop{}h3(f5,e6,d7,c8) 2.\knight{}f3+ \king{}e4 3.\bishop{}d3\mate }

Центральный вариант демонстрирует, почему необходимо было уходить на f1: \textbf{ 1... \bishop{}e2! 2.\king{}xf2! \king{}c4 3.\king{}e3\mate}. Играет королевская батарея!

\begin{center} 
 \begin{tabular}{ c c }
\textbf{№ 31. Мушет Григорий} & \textbf{№ 32. Димов Дмитрий} \\
\small{} & \small{}\\
\small{} & \small{}\\
\chessboard[
\diagramsize,
setfen=4b3/R2N2K1/5p2/2P2P2/Pb1k1P2/8/1p2n3/8,
label=false,
showmover=false] & 
\chessboard[
\diagramsize,
setfen=8/2p1Qp2/K1k1bB1r/4P3/3q3p/7R/8/8,
label=false,
showmover=false] \\
\textbf{H\mate{}3 3 решения} & \textbf{+} 
 \end{tabular}
\end{center}

№ 31  \begin{enumerate*}[label={\alph*)}]
\item \textbf{1.\bishop{}d2 \knight{}xf6 2.\bishop{}xf4 \rook{}d7+ 3.\king{}e5 \rook{}d5\mate }
\item \textbf{1.\bishop{}xc5 \rook{}c7 2.\bishop{}d6 \rook{}c4+ 3.\king{}d5 \knight{}b6\mate }
\item \textbf{1.\king{}c3 \knight{}e5 2.\knight{}d4 \rook{}d7 3.\knight{}c2 \rook{}d3\mate }
\end{enumerate*}

№ 32 Исправление: \textbf{1. \queen{}e8+ \bishop{}d7 2. \queen{}a8+ \king{}c5 3. \queen{}a7+ \king{}d5 4. \queen{}xd4+ \king{}xd4 5. \rook{}xh4+! \rook{}xh4 6. e6+ \king{}c5 7. exd7 \king{}c6 8. d8=\knight{}+! \king{}d5 9. \bishop{}xh4 } с победой

