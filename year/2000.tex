\chapter{Год 2000}
\textbf{Судья: Гордиан Юрий (Украина)}

\begin{tabularx}{\textwidth}{l l r}
Место & Участник & Баллы \\
\hline
1 место & Чиканов Николай & 4,9 \\
2 место & Иванов Альберт & 4,7 \\
3 место & Воронов Анатолий & 4,1 \\
4 место & Чебанов Николай & 4 \\
\end{tabularx}

\begin{center} 
 \begin{tabular}{ c c }
\textbf{№ 10. Чиканов Николай} & \textbf{№ 11. Иванов Альберт} \\
\small{V чемпионат мира} & \small{Уральский проблемист}\\
\small{1972 (переработка)} & \small{1999}\\
\chessboard[
\diagramsize,
setfen=B3n3/5R2/4p2q/r1p1p3/3pNP2/1R2Nk2/1KPp4/5bbn,
label=false,
showmover=false] & 
\chessboard[
\diagramsize,
setfen=3K3B/2nR4/4k3/2Pr4/2r4n/3Q4/B6b/8,
label=false,
showmover=false] \\
\textbf{H\mate{}2   7 решений} & \textbf{\mate{}2} 
 \end{tabular}
\end{center}

№10 \textbf{1.\bishop{}e2 \knight{}c3+ 2.\king{}xe3 \knight{}d1\mate{}; 1.\bishop{}xe3 \rook{}b7 2.\king{}xe4 \rook{}a7\mate{}; 1.de3 \rook{}b7 2.\king{}xe4 \rook{}b4\mate{}; 1.\rook{}b5 \knight{}d5+ 2.\king{}xe4 \knight{}b6\mate{}; 1.\queen{}xf4 \rook{}b7 2.\king{}xe4 \rook{}a7\mate{}; 1.\queen{}g5 \knight{}f6+ 2.\king{}xf4 \knight{}h5\mate{}; 1.\knight{}c7 \knight{}f5+ 2.\king{}xf4 \knight{}xh6\mate{}}

№11 Задача представляет тему Яновчика: у белых связанная фигура, у черных полусвязка. После вступительного хода черные уходят одной фигурой с линии полусвязки, в результате чего матует связанная белая фигура: \textbf{1. \queen{}h7! $\sim$ 2.\queen{}f7\mate{}; 1. ... \rook{}f4 2.\rook{}d6\mate{}} -- тема Яновчика. \textbf{1. ... \rook{}xd7+ 2.\queen{}xd7\mate{}; 1. ... \rook{}f5 2.\queen{}e7\mate{}}

\begin{center} 
 \begin{tabular}{ c c }
\textbf{№ 12. Воронов Анатолий} & \textbf{№ 13. Чебанов Николай} \\
\small{Stilistica} & \small{Уральский проблемист} \\
\small{2000} & \small{2000} \\
\chessboard[
\diagramsize,
setfen=6b1/8/8/p1Kp4/8/2P5/2Nk4/8,
label=false,
showmover=false] & 
\chessboard[
\diagramsize,
setfen=6k1/6p1/2N3K1/p5P1/2p5/8/8/8,
label=false,
showmover=false] \\
\textbf{=} & \textbf{=} 
 \end{tabular}
\end{center}

№12

№13
