Данный сборник шахматной композиции был мечтой Иванова Альберта Федотовича, старейшины композиторского цеха Молдовы.

В основу сборника легли результаты чемпионатов республики Молдова по составлению шахматных задач. Материалы были предоставлены  Ивановым. Как будет видно, в чемпионатах принимали участие композиторы с задачами из различных жанров. Это связано с тем, что в Молдове малое количество людей интересуется таким видом искусства как шахматная композиция и, соответственно, еще меньше людей составляет задачи и этюды.

По правилам чемпионатов, участники могли участвовать с любой своей задачей (этюдом), опубликованной или оригинальной, не участвовавшей ранее в чемпионатах. Победитель предыдущего чемпионата имел право предоставить 2 произведения. Для судейства привлекались композиторы из других стран, за исключением последнего чемпионата, в котором присуждением занимались сами участники.

Позиции, при подготовке сборника, прошли компьютерную проверку. Часть задач (и призовые тоже) оказалась с дефектами. Я приложил усилия по сохранению задач, однако часть из них так и не смог исправить. Однако, эти позиции всё равно входят в сборник, сразу по нескольким мотивам: во-первых, в целях сохранения истории; во-вторых, с целью демонстрации представляемых ими идей; в-третьих, в надежде, что читатели сборника предложат свои исправления.

Надеюсь, что данная брошюра окажет своё влияние на развитие шахматной композиции в Молдове. Я пытался писать комментарии к решениям задач на доступном для любителей шахматной композиции уровне. 
